The manuscript has been reviewed by an expert reviewer. The reviewer's overall assessment on the quality of the manuscript for publication consideration is generally positive. The reviewer recommends minor revisions.

With the positive evaluation, the reviewer also raises some important questions and offer useful suggestions to improve the quality of the manuscript further. The reviewer's comments include: (1) need for re-organization of the contents of some parts; (2) need to technically strengthen the manuscript, especially by discussing the human activity (in addition to environmental disturbance); and (3) need to strengthen the presentation of the interaction between hydrology and biogeochemical cycle.

I have read the manuscript, in light of the review comments and my own knowledge of the topic. I share the assessment by the reviewer. The manuscript addresses an important problem within the context of the Special Issue and presents a very interesting perspective, with a comprehensive review. However, the manuscript still requires some improvements to be positively considered for publication. The reviewer's comments should certainly help in this regard.

In view of these, I recommend minor revisions.




Reviewer #1: The manuscript submitted by Dr. Chen reviewed the field observation and up-to-date modeling related to watershed water quantify and quality, especially the effect of environmental perturbations. Generally, it is a outstanding and timely review, given that our earth are experiencing large natural and anthropological disturbances and it is necessary to integrate big data and modeling to predict potential challenges. I personally learn a lot from this review. However, I also have some suggestions to revise.    

1) to re-arrange parts of the manuscript. For example, this manuscript introduce comprehensively different monitoring systems and models, but the internal integration between them is not explicated. the authors should discuss more about how the monitoring system or data stimulate or improve modeling development, and how modeling help to understand the monitoring results. Although the authors spend a lot of time to review the monitoring system and modeling and mentions the importance of developing monitoring system based on modeling demands, it is necessary to discuss the how monitoring and modeling integrate. The authors could at least give some cases in this integration and point out the shortcomings.

2) to introduce more about environmental disturbance. The current manuscript mainly talked about extreme weather. However, human activity such as urbanization and deforestation have huge effect on water quantify and quality in watershed. The authors should include and discuss how these human activities affect watershed hydro-bgc and how to conceptualize and quantify the effects in modeling. For example, nitrate and other contaminants related to human activity significantly affect water quality people care, but they are simplify discussed in this manuscript. By contrast, DOM is less related to water quality but is heavily discussed.    

3) to replace the "reactive transport modeling" in the title. reactive transport modeling is not the most important key word in this manuscript, although it is very important for predicting water quality. From my perspectives, hydrology modeling may be more critical for predicting watershed-scale water quality. Besides, the authors also do not put much effort to highlight reactive transport modeling in the context. It is misleading to put "reactive transport modeling" in the title.

4) Fig. 1 only presents a weak interaction between hydrology and bgc, and shows a loose link across different scales. The authors should specify this conceptual model and make it strongly deliver the info how hydrology and bgc interact and how processes at different scales link.