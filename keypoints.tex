\documentclass[preprint,review, 12pt]{elsarticle}

%% Use the option review to obtain double line spacing
%% \documentclass[preprint,review,12pt]{elsarticle}

%% Use the options 1p,twocolumn; 3p; 3p,twocolumn; 5p; or 5p,twocolumn
%% for a journal layout:
%%\documentclass[final,1p,times]{elsarticle}
%% \documentclass[final,1p,times,twocolumn]{elsarticle}
%% \documentclass[final,3p,times]{elsarticle}
%% \documentclass[final,3p,times,twocolumn]{elsarticle}
%% \documentclass[final,5p,times]{elsarticle}
%% \documentclass[final,5p,times,twocolumn]{elsarticle}

%% if you use PostScript figures in your article
%% use the graphics package for simple commands
%% \usepackage{graphics}
%% or use the graphicx package for more complicated commands
\usepackage{graphicx}
%% or use the epsfig package if you prefer to use the old commands
%% \usepackage{epsfig}

%% The amssymb package provides various useful mathematical symbols
\usepackage{amssymb}
%% The amsthm package provides extended theorem environments
%% \usepackage{amsthm}

%% The lineno packages adds line numbers. Start line numbering with
%% \begin{linenumbers}, end it with \end{linenumbers}. Or switch it on
%% for the whole article with \linenumbers after \end{frontmatter}.
\usepackage{lineno}




%% natbib.sty is loaded by default. However, natbib options can be
%% provided with \biboptions{...} command. Following options are
%% valid:

%%   round  -  round parentheses are used (default)
%%   square -  square brackets are used   [option]
%%   curly  -  curly braces are used      {option}
%%   angle  -  angle brackets are used    <option>
%%   semicolon  -  multiple citations separated by semi-colon
%%   colon  - same as semicolon, an earlier confusion
%%   comma  -  separated by comma
%%   numbers-  selects numerical citations
%%   super  -  numerical citations as superscripts
%%   sort   -  sorts multiple citations according to order in ref. list
%%   sort&compress   -  like sort, but also compresses numerical citations
%%   compress - compresses without sorting
%%


\journal{Journal of Hydrology}

\begin{document}
\begin{frontmatter}

%% Title, authors and addresses

%% use the tnoteref command within \title for footnotes;
%% use the tnotetext command for the associated footnote;
%% use the fnref command within \author or \address for footnotes;
%% use the fntext command for the associated footnote;
%% use the corref command within \author for corresponding author footnotes;
%% use the cortext command for the associated footnote;
%% use the ead command for the email address,
%% and the form \ead[url] for the home page:
%%
%% \title{Title\tnoteref{label1}}
%% \tnotetext[label1]{}
%% \author{Name\corref{cor1}\fnref{label2}}
%% \ead{email address}
%% \ead[url]{home page}
%% \fntext[label2]{}
%% \cortext[cor1]{}
%% \address{Address\fnref{label3}}
%% \fntext[label3]{}

\title{Integrating Field Observations and Reactive Transport Modeling to Predict Watershed Water Quality under Environmental Perturbations}

%% use optional labels to link authors explicitly to addresses:
%% \author[label1,label2]{<author name>}
%% \address[label1]{<address>}
%% \address[label2]{<address>}

\author{Xingyuan Chen\corref{cor1}\fnref{label1}}
\ead{Xingyuan.Chen@pnnl.gov}
\fntext[label1]{Pacific Northwest National Laboratory, Richland, WA, United States}
\author[label1]{Raymond Mark Lee}
\author{Dipankar Dwivedi\fnref{label2}}
\fntext[label2]{Lawrence Berkeley National Laboratory, Berkeley, CA, United States}
\author[label1]{Kyongho Son}
\author[label1]{Yilin Fang}
\author[label1]{Xuesong Zhang}
\author[label1]{Emily Graham}
\author[label1]{James Stegen}
\author{Joshua B. Fisher\fnref{label3}}
\fntext[label3]{Jet Propulsion Laboratory, California Institute of Technology, Pasadena, CA, United States}
\author{David Moulton\fnref{label4}}
\fntext[label4]{Los Alamos National Laboratory, Los Alamos, NM, United States}
\author[label1]{Timothy D. Scheibe}



%\address{Atmospheric Science and Global Change Division, P.O. Box 999, Pacific Northwest National Laboratory, Richland, WA}

\section {highlights}
\item We need to understand how environmental perturbations including extreme events propagate through watershed systems;
\item Current watershed models do not fully reflect current status of process understanding;
\item A national integrated watershed modeling capability is needed to address water missions across multiple agencies.

\end{frontmatter}
\end{document}